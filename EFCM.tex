\documentclass[11pt]{article}
\usepackage[backend=biber, style=numeric, url=true, doi=true]{biblatex}
\addbibresource{EFCM.bib}
\usepackage{url}
\usepackage{breakurl}
\usepackage[colorlinks=true, linkcolor=black, citecolor=blue, urlcolor=blue]{hyperref}

\title{Elements of Economics, Finance, and Computational Mathematics}
\date{\today}
\author{BRICIU, RADU-ȘTEFAN}
\begin{document}
\nocite{*}
\maketitle
\newpage
\begin{abstract}
	\noindent
	We examine an emerging pedagogical realm in which the importance of three major disciplines are considered in synchronicity. The aim is to understand the coherence of an interdisciplinary science formed around modern Economics, Finance, and Computational Mathematics. We recognize the rapidly evolving progress in data encoding techniques and contemplate economic, financial, and societal phenomena that may arise from technology evolving at increasing rates.
\end{abstract}
\newpage
\tableofcontents	
\newpage

\section{Intentions}
I am uncertain around the bounds and validity of my knowledge and feel the need to somehow organize my honestly informed view of what I deem to be relevant areas of expertise that are possibly within reasonable grasp. Therefore, I decided to start archiving my thoughts in the form of prose accompanied by scrutinized literature to build my understanding of the world. I believe that all learning is a collective matter and cordially invite those who are interested in the topic to join this discussion.
\newpage

\section{Introduction}
Throughout this paper, we intend to collect all relevant literature in attempting the justification of every argument made, especially those where objectivity is not feasible. Naturally, there will be conflicting opinions in the matter, but it is these very conflicts that allow for constructive debate when governed correctly by every individual involved. We draw from René Descartes' \textit{Discourse on the Method} \cite{renedescartes_2008_a} in saying \textit{governed}. Specifically, we impose the rationalization of any uncertain context in the most objective manner available, invariably \textit{seeking truth in the sciences} while scrupulously ensuring the veracity of our reasoning in matters which may appear either objective or subjective.

\section{Declaration of opinions}

We will preface this paper with a few notions believed to be true, honest, and scientifically informed to the best of our abilities.

The first fundamental axiom we consider is that recent evolutions in cryptography~\cite{ernstberger_2024_do}~\cite{kuznetsov_2024_enhanced}~\cite{nadler_2023_tornado}~\cite{shahlaatapoor_2023_vss} create new possibilities in many financial, legal, and economic processes allowing for the properties of full transparency and trust-less execution. Drawing from the famous economic school of thought founded by Adam Smith~\cite{smith_1776_the}, we make an inherent connection between economic and social prosperity. We also borrow the illustrious concept of \textit{efficient market hypothesis} as famously described by Fama and Samuelson~\cite{delcey_2019_samuelson} to justify the veracity of market convergence to optimal results given sufficient coverage and volume. We recognize contemporary technology as a potentially pernicious solution to economic processes and their corruption by malicious intent. Semantic junctures between various meanings of the word \textit{corrupt} lend themselves perfectly to our study for one principal reason: we claim that a mathematically non-corruptible informational process is the ideal tool for transitioning vital financial processes to a transparent framework designed to protect against malicious human intent (i.e., \textit{corruption}).

\section{Recent computational developments}
\subsection{Zero Knowledge Proofs}
\subsection{Machine learning in numerical methods}


\section{Decentralization and Ethics}
\subsection{Recent case studies}
\subsubsection{Kleros~\cite{zhuk_2023_applying}}
\subsubsection{Tornado Cash~\cite{nadler_2023_tornado}~\cite{pertsev_2019_tornado}}
\subsection{Perspectives}


\section{Hypotheticals}
\subsection{Macroeconomic trend prediction using live shipping data}
\subsection{Decentralization for internal and external policy}
\subsection{Commodities markets and their decentralization}
\subsection{FX option markets and their decentralization}
\subsection{Commerical legal services and their decentralization}

\newpage
\printbibliography
\addcontentsline{toc}{section}{References}
\end{document}