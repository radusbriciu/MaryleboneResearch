\documentclass[11pt]{article}
\usepackage[backend=biber, style=numeric, url=true, doi=true]{biblatex}
\addbibresource{EFCM.bib}
\usepackage{url}
\usepackage{breakurl}
\usepackage[colorlinks=true, linkcolor=black, citecolor=blue, urlcolor=blue]{hyperref}
\usepackage{graphicx}
\graphicspath{{images/}}


\title{[Elements of Economics, Finance, and Computational Mathematics]}
\date{\today}
\author{Radu Briciu}
\begin{document}
\nocite{*}
\maketitle
\newpage
\begin{abstract}
	\noindent
	We examine an emerging pedagogical realm in which the importance of three major disciplines are considered in synchronicity. The aim is to understand the coherence of an interdisciplinary science formed around modern Economics, Finance, and Computational Mathematics. We recognize the rapidly evolving progress in data encoding techniques and contemplate economic, financial, and societal phenomena that may arise from technology evolving at increasing rates.
\end{abstract}
\vspace{5em}
Join the discord: https://discord.gg/HHStaC4h

\newpage
\tableofcontents	
\newpage

\section{Introduction}
I am developing this paper on my own in the hopes it becomes a larger group project. I believe all learning is a collective matter and cordially invite all people interested into the discussion of such a portfolio of ideas.

\section{Recent computational developments}
\subsection{Zero Knowledge Proofs}
Zero Knowledge proofs~\cite{ernstberger_2024_do} are a familiar cryptographic concept with recent applicability to scaling financial circuitries~\cite{leethorp_2022_fnet}.
\subsection{Machine learning in numerical methods}
Machine learning methods allow for greater accuracy in all manner of numerical computation and model interpretation tasks. Interpretability of models allows for easier and more accurate volatility modelling~\cite{beck_2019_machine}~\cite{kirenz_2022_using}~\cite{parr_2021_partial}~\cite{yuan_2024_deep}.

\subsection{Artificial General Intelligence (AGI)}
\subsubsection{OpenAI}
\subsubsection{Google}
\subsubsection{GitHub}

\section{Internet tools}
With technological hardware becoming increasingly accessible, powerful internet tools emerge that may require immediate attention or regulation. Others however, are useful in pedagogical applications.
\subsection{Tornado Cash~\cite{nadler_2023_tornado}~\cite{pertsev_2019_tornado}}
Tornado Cash is a peculiar blockchain smart contract protocol that allows users to send and receive cryptocurrency payments completely anonymously. The protocol has purportedly been flagged by US authorities leading to very little utilization.
\subsection{TinyURL~\cite{a2019_tinyurlcom}}
TinyURL advertises a useful link shortening tool with added security features along a paid subscription. Modern cryptographic links are long and may seem intimidating especially when domain names change rapidly.
\subsection{MyBib~\cite{mybib_2018_mybib}}
MyBib is a powerful tool for understanding how BibTeX files work, bringing new dimensions to independent research.
\subsection{ProtonMail~\cite{kobeissi_2018_an}}
ProtonMail is a popular emailing service based in Switzerland claiming to offer end-to-end encryption. The existence of such services call into question the feasibility of further scaling to internet privacy infrastructure. An analysis~\cite{kobeissi_2018_an} on the service's security is offered by Kobeissi.
\subsection{The Wikimedia Foundation~\cite{a2018_wikimedia}}
The Wikimedia Foundation emerged as a financial supporter of the Wikipedia project, allowing independent researchers to support the movement around freedom of scientific information.
\subsection{Data Annotation~\cite{data}}
Data Annotation is a website that purportedly pays its users to redact and assess AI chatbot prompts and responses.
\subsection{Online Museums}
Berkeley, University of California publishes art photography online~\cite{a2024_library} for universal visual access to painting, scriptures, documents, sculptures, trinkets, and other art history relics.
\subsection{Subdial~\cite{a2023_subdial}}
Subdial is an online marketplace where watches are verified per the listing specification.
\subsection{EconViz~\cite{econvizorgmacroeconomicvisualizations_2025_econvizorg}}
EconViz is a free online tool that helps visualize classical economic theory.
\subsection{The Political Compass~\cite{thepoliticalcompass_2018_the}}
The political compass is an interesting, seemingly satirical website that might be the perfect thought experiment to generate the next big behavioural finance and economics study in the context of LLMs and algorithmic trading.
\subsection{Bitcoin Core~\cite{bitcoin}}
Bitcoin core is a multiplatform solver that mines tokens for its user once the system running it is synchronized to the Bitcoin network.
\subsection{TeXstudio}
TeXstudio is a powerful open-source LaTeX EDU, allowing for easy access to PDF file construction.

\section{\textit{The OS wars}}
\subsection{Environment Intractability}
\subsubsection{Entropy Pooling~\cite{vorobets_2024_portfolio}}
\subsubsection{Stochastic Volatility~\cite{heston_1993_a}~\cite{briciu_2024_estimating}}
\section{Quantitative trading strategies}
\subsection{Macroeconomic trend prediction using live shipping data}
\section{Blockchain security and pipeline transparency}
\subsection{Commodities markets}
\subsection{FX option markets}
\subsection{Commerical legal services}
\section{Engineering concepts}
\subsection{Integrating solar panels into protected forestry domain}
\section{Modern Tactics}
\subsection{Artificial Scarcity}
The role of \textit{hype} in parsimonious production and consistent sales.
\subsection{Herding}
Huang et. al. \cite{huang_2015_herd} have described an irrational herd behaviour in financial markets for over two decades. Many studies have arose from studying seeming arbitrary volatility and correlations in equity markets. With LLMs largely available, one might be able to construct various algorithmic trading strategies based on the herding of volatile assets.
\newpage
\printbibliography
\addcontentsline{toc}{section}{References}
\end{document}