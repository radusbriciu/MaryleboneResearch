\documentclass[11pt]{article}
\usepackage[backend=biber, style=numeric, url=true, doi=true]{biblatex}
\addbibresource{EFCM.bib}
\usepackage{url}
\usepackage{breakurl}
\usepackage[colorlinks=true, linkcolor=black, citecolor=blue, urlcolor=blue]{hyperref}

\title{Elements of Economics, Finance, and Computational Mathematics}
\date{\today}
\author{BRICIU, RADU-ȘTEFAN}
\begin{document}
\nocite{*}
\maketitle
\newpage
\tableofcontents	
\newpage
\begin{abstract}
	\noindent
	We examine an emerging pedagogical realm in which the importance of three major disciplines are considered in synchronicity. The aim is to understand the coherence of an interdisciplinary science formed around modern Economics, Finance, and Computational Mathematics. We recognize the rapidly evolving progress in data encoding techniques and contemplate Economic, Financial and Societal risks that can arise from technology evolving at increasing rates.
\end{abstract}
\newpage

\section{Intentions}
I often find myself bogged down in a plethora of contemplative thoughts that leave me unproductive and depressed. I am uncertain around the bounds and validity of my knowledge and feel the need to somehow organize the most relevant areas of expertise that are possibly within reasonable grasp. Therefore, I decided to start archiving my thoughts in the form of prose accompanied by scrutinized literature to build my understanding of the world. I believe that all learning is a collective matter and cordially invite those who are interested in the topic to join this discussion.
\newpage

\section{Introduction}
Throughout this paper, we intend to collect all relevant literature in attempting the justification of every argument made, especially those where objectivity is not feasible. Naturally, there will be conflicting opinions in the matter, but it is these very conflicts that allow for constructive debate when governed correctly by every individual involved. We draw from René Descartes' \textit{Discourse on the Method} \cite{renedescartes_2008_a} in saying \textit{governed}. Specifically, we impose the rationalization of any uncertain context in the most objective manner available, invariably \textit{seeking truth in the sciences} while scrupulously ensuring the veracity of our reasoning in matters which may appear either objective or subjective.
\newpage


\newpage
\printbibliography
\addcontentsline{toc}{section}{References}
\end{document}